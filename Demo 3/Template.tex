\documentclass[hidelinks,a4paper,12pt]{article}
\addtolength{\oddsidemargin}{-1.cm}
\addtolength{\textwidth}{2cm}
\addtolength{\topmargin}{-2cm}
\addtolength{\textheight}{3.5cm}
\newcommand{\HRule}{\rule{\linewidth}{0.5mm}}
\makeindex

\usepackage{longtable}
\usepackage[pdftex]{graphicx}
\usepackage{makeidx}
\usepackage{hyperref}
\hypersetup{
    colorlinks=true,
    linkcolor=blue,
    filecolor=magenta,      
    urlcolor=cyan,
}


% define the title
\author{Men-at-Work}
\title{ OnlyRugby Mobile Application User Manual}
\begin{document}
\setlength{\parskip}{6pt}

% generates the title
\begin{titlepage}

\begin{center}
% Upper part of the page       
\includegraphics[width=1\textwidth]{./images/up-logo.jpg}\\[0.4cm]    
\textsc{\LARGE Department of Computer Science}\\[1.5cm]
\textsc{\Large COS 301 - Software Engineering}\\[0.5cm]
% Title
\HRule \\[0.4cm]
\includegraphics[width=0.05\textwidth]{./images/logo.jpg} 
{ \huge \bfseries Amazon}
\includegraphics[width=0.05\textwidth]{./images/logo.jpg}\\[0.4cm] 
{ \huge \bfseries Network Visualizer}\\[0.4cm]
{ \huge \bfseries Demo 3}\\[0.4cm]
\HRule \\[0.4cm]
% Author and supervisor
\textsc{\Large Not-Like-This}\\[0.5cm]
\begin{minipage}{0.4\textwidth}
\begin{flushleft} \large
\emph{Authors:}
\end{flushleft}
\end{minipage}
\begin{minipage}{0.4\textwidth}
\begin{flushright} \large
\emph{Student number:}
\end{flushright}
\end{minipage}

\begin{minipage}{0.4\textwidth}
\begin{flushleft} \large
Jedd {Schneier}
\end{flushleft}
\end{minipage}
\begin{minipage}{0.4\textwidth}
\begin{flushright} \large
\emph{}
u13133064
\end{flushright}
\end{minipage}

\begin{minipage}{0.4\textwidth}
\begin{flushleft} \large
Daniel {King}
\end{flushleft}
\end{minipage}
\begin{minipage}{0.4\textwidth}
\begin{flushright} \large
\emph{}
u13307607
\end{flushright}
\end{minipage}

\begin{minipage}{0.4\textwidth}
\begin{flushleft} \large
Muller {Potgieter}
\end{flushleft}
\end{minipage}
\begin{minipage}{0.4\textwidth}
\begin{flushright} \large
\emph{}
u12003672
\end{flushright}
\end{minipage}

\vfill
% Bottom of the page
{\large \today}
\end{center}
\end{titlepage}
\footnotesize
%\input{declaration_of_originality.tex}
\normalsize


\pagenumbering{roman}
\tableofcontents
\newpage
\pagenumbering{arabic}

\newpage
\section{Vision} 
The Network Visualizer is intended to be used by registered Amazon Web Services (AWS) users. The visualizer is primarily aimed at consumers of AWS, in order to provide a simple and clear representation of the various networks' structures.

In order to access the visualizer, the user must first submit their AWS password and secret password. The visualizer then attempts to access the server, using the provided passwords. If this is successful, the visualizer will scan the specified network and log the nodes (such as instances and VPC's) and their relationships. It uses this information to construct a tree-esque representation of the network.

Using this representation, it is then translated into HTML. Making use of the vis.js library, the structure of the network is presented in a clear, visual hierarchial structure. The page also allows the user to specify which region they wish to be scanned and represented.

\section{Scope}
\begin {itemize}
	\item The user requires a valid Amazon Web Services account (or access to one), in order to make use of the visualizer.
	\item The user requires an internet capable device and a modern browser to access the visualizer's page.
\end{itemize}
\newpage

\section{Architectural Requirements}

	\subsection{Architecture Scope}
	Layered Architecture: 
	

	\subsection{Quality Requirements}
		\subsubsection {Critical}
		\subsubsection {Important}
		\subsubsection {Nice To Have}
	\subsection{Architectural Constraint}
	
\section{Architectural Design}
Layered Architecture: 

Our core architectural design follows a standard Client-Applciation-Server layered architecture. As seen below, the "Server" (not to be confused with the Application server) is the AWS server we connect to through their API. The product does not persist to only reads from it and is we have no controll over its functionality it will be left for an intergratio requirment and will not be disccues furhter in this section. Additional information on the AWS ssytem and their API can be found here https://aws.amazon.com/documentation/ . Previously we useda  bridging layer ,ADAPTER, to join the Application and Server layer, but  it has sesne been incorporated into the Scanner to reduce communciation overhead and facilitate the threading tactics. The main backend of the system lies on the Application layer, which is broken down into the follwoing components:
\begin{enumerate}  
					\item Scanner: Most important component, constructs a scan based on differnt priorities. Forms a Producer-Consumer relationship with the Visualizer. Once launced it will continue to produce network trees(see below) for the buffer until it recieves a pause/stop command, or completes a scan of the network. 
					\item Composite:  Following the composite design pattern, this object represents the hierarchial Network tree, either in part in whole.  
					\item Threaded sub-Scanners: Each logial part of an AWS network has an assigned threaded scanner to it. Depeding on what is a priority to scan first the scanner will launcha number of threads for each part wich in turn will launch its on sub scan threads based on perfromance requirments. Each threaded SubScanner will scann 100 of its children,place in buffer and then contineu to scan another 100. 
					\item Smart buffer: The link between the Visualizer and the Scanner as wel las the "brain" of the system. The individual scanner threads wil ladd to the buffer as soon as they have finished constructinga tree then go back to scanning. The trees arrive disjoint and posisbly overlap. The smart buffer constructs the entire network tree from the disjoint trees it recieves accurately. It stores the moat upto date construct for the Visualizer.
				\item REST API and Server: The REST server joins the backend to the front end. See architecture below.
				\end{enumerate}

The client layer consists of a Single page Application and the Visualizer. The SPA is tee user interface with all the scanning options and controll. and the visualized graph and scanned information. The Visualizer polls on the REST server requesting latest tree from the smart buffer, rendering it and presenting it to the user.

RESTFUL Design:

User interaction is mapped from the user interface onto diffennt API calls on the server. Each call launches a backend method to fullfill the request. The API calls are:
\begin{enumerate}  
					\item POST, ConnectToAccount: Takes the users acces and private key and creates a conenction to AWS for the remainder of the session .
					\item GET, ScanNetwork: Launches a scan of the entire network, Visualizer polls on results as they stream in. 
					\item GET, StopScan: Cancels current scan.
					\item GET, PauseScan: Pauses current scan.
					\item GET, ResumeScan:Resumes current scan Visualizere polls on buffer.
					\item GET, ScanFrom: Performs a subscan starting from the given point and scanning around its general vicinity.
\item GET, ScanUP: Scans the next logical area above what the ScanFrom scanned.
 \item GET, ScanDown: Scans the next logical area below what the ScanFrom scanned.
				\end{enumerate}

Threading Tactic:
Each network component Region,VPC,Subnetwork,Instance has a threaded scanner associated to it. it si up to the Scanner depending on what the user is searching for to construct the scan from this subscanners.
By defualt The scanner will thread a RegionScanner for each region and each one will have one scanner for the other parts. However if the scan is narrowed then scanner will thread on the most cost and performance effective way to compelte the scan . For example if the scan was to only scan a single VPC then other Regions would not need to be scanned and thus only one RegionalScanner will be launced, one VPCScanner will be launced and it with thread on collections of its subnetworks. We trying to keep number of threads needed to a mninimum, but they are invaluabel to speeding up the scanning.  



\newpage
\section{Functional Requirements}
	This section specifies the functional specifications for the AWS Network Visualizer system. It defines the user-system interaction and relationship between users and the product. It will provide the expected functionality for all user cases as well as the activity processes for the system.
	
		\subsection{Description of AWS}
			Amazon web services provide a cloud based service for hosting a clients network. There is a lack of information regarding ones network, specifically the logical representation on the system for the client to make sense of their network. The network visualizer aims to improve clients understanding of their own network, how AWS works and possible insights in the managing of their network.
		\subsection{Required Functionality}
			The system must be able to:
				\begin{enumerate}  
					\item Be accessible to registered, valid AWS customers.
					\item Scan the networks located in different regions. 
					\item Provide an interactive hierarchical and visual representation of the networks.
					\item Give additional information on the network statistics, construction, etc.
					\item Provide a clear image of the clients' virtual networks.
					\item Improve the AWS clients' experience.
				\end{enumerate}
		\subsection{Use case Prioritization}
		
		\subsection{Use case/Service contracts}
		\begin{tabular}{ | p{3cm} | p{4cm} | p{4cm} | p{4cm} |}
			\hline
			Use Case & Pre Condition & Post Condition & Description \\ \hline
			
			Viewing the Hierarchy & The application must be connected to the server and able to read in network data. The server must be active and have acess to AWS.& The web page continuosly updates with the new data being loaded. & This use case forms the core of the project, as the primary purpose of the application is to visualise the structure of the virtual network.\\ \hline
			
			Select a region 	& The application must be connected to the server and able to read in network data. The server must be active and have acess to AWS.& The previous region's visualisation is replaced by the selected region. & There are a number of AWS regions. For simplicity's sake, the visualiser only visualises one at a time, but allows for switching betweeen them.\\ \hline
			
			Zooming in/out. Moving the hierarchy	& The application must be connected to the server and able to read in network data. The server must be active and have acess to AWS.& The hierarchy's position or level of zoom is altered. & Since the visualised network may be large, the user can move it about and zoom in, for a better view.\\ \hline
			
			Clicking a node.& The application must be connected to the server and able to read in network data. The server must be active and have acess to AWS.& The selected node is highlighted. Node related information is displayed below the hierarchy. & Each node in the network has a number of attributes that can provide valuable information. This way, the information can be displayed in a neat manner.\\ \hline
			
			Hovering over a node.& The application must be connected to the server and able to read in network data. The server must be active and have acess to AWS.& Edges connecting nodes of different levels are rendered invisible. Edges connecting nodes on the same level are made vidible. & It is possible for nodes on the same level of the network to have relationships. They are normally hidden, in order to present a cleaner hierarchy. \\ \hline
		\end{tabular}
		
	
\newpage
\section{Specification Update}
























\section{Using the system}
The functionality of the web page will be spread between the following use cases:

	\subsection{Login}
	
			This is the first web page of the visualizer. The user is presented with two fields: One for the user's normal AWS passoword and another for their secret AWS password. The user then enters the requested details and the server is queriedm to verify the user's identity. Only when the correct passords are supplied, will the user be able to continue to the visualizer.
		
    		\includegraphics[width=0.9\textwidth]{./images/Login1.png}
    		
    		If the user provides incorrect information, they presented with a message indorming them of this ocurrence.
    		
		  	\includegraphics[width=0.9\textwidth]{./images/Login2.png}

	\newpage	
	\subsection{Visualizer}
		This use case pertains to the network visualization. The network is presented in a hierarchial manner, to make traversing the tree easier.
		
		\includegraphics[width=0.9\textwidth]{./images/Visualizer1.png}
	\subsection{Regions}
		The dropdown menu in the left column displays the active region's name. The active region is the one that is shown in the visualizer.
	
		\includegraphics[width=0.9\textwidth]{./images/Visualizer2.png}
		
		Clicking the menu will display a list of the official AWS regions. Clicking one of the options will set the current region.
			
		\includegraphics[width=0.9\textwidth]{./images/Visualizer3.png}
		
		\newpage
		Clicking the "Load selected region" button sends a request to the server, which will then load the selected region and return a web page with the selected network representation
		
		\includegraphics[width=0.9\textwidth]{./images/Visualizer4.png}

	\subsection{Zooming}
	Hovering the mouse over the visualizer's window and rolling the mouse wheel will change the zoom of the visualizer. 
	
	\includegraphics[width=0.9\textwidth]{./images/Visualizer5.png}
	\newpage

	\subsection{Moving}
	Clicking and dragging inside the visualizer's window will move the image.
	
	\includegraphics[width=0.9\textwidth]{./images/Visualizer6.png}

	\subsection{Selection}
	Clicking on a node will highlight the selected node and its relationships.
	\newline
		\includegraphics[width=0.9\textwidth]{./images/Visualizer7.png}
		
		\newpage
	\subsection{Information}
	Hovering the mouse over a node will display a small box, specifying its name and the number of VPC's it has.
	
	\includegraphics[width=0.9\textwidth]{./images/Visualizer8.png}
	\newline
	

\end{document}
